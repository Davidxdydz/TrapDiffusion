\begin{document}
\title{Surrogate Solver for Trap Diffusion of Hydrogen Isotopes (Working title)}
\author{David Berger}


\begin{abstract}
    Bla Bla
\end{abstract}

\section{Introduction}
\paragraph{Requirements}
- < 50µs execution
- < 1\% max Error
- conservation of mass up to machine resolution
\section{Theory}
\subsection{Single Isotope, Single Occupation}
\subsection{Multi Isotope, Multi Occupation}
\section{Method}
- General: Model params, inputs, outputs
\subsection{Random Search}
- Single Occupation Single Isotope
- Relu vs Leaky Relu
\subsubsection{Requirement 1: Execution < 50µs}
- Tradeoff size <-> accuracy/speed
- Selection for small models

\subsubsection{Requirement 2: Max Error}

\subsubsection{Requirement 3: Mass Conservation}

\subsection{Generalization to Multi Isotope Multi Occupation case}
- Needs more params, duh
- activations and stuff still works

\section{Results}
- This is my model, with those hyperparams and this loss
- SOSI fixed
- SOSI random
- MOMI fixed
- MOMI random?
\subsection{Robustness}
% - Numerical/ analytical solver can fail, this model not xddd
The configurations for which the network yields the worst results mainly show problems in the dataset and confirm the robustness of the learned function.

The ode solver used to generate the ground truth (\ref{data}) can fail to give a correct solution for some rare configurations (~1 in 50,000). 
Whilst the solutions still looks plausible and adhere to conservation of mass, the network can't seem to learn these configurations and predicts deviating results.
After decreasing the solvers relative tolerances by several orders of magnitude, the numerical solution now agrees with the network's prediction.

\includegraphics[width=\textwidth]{figures/MOMI_fixed_normalized_worst_case_fixed.pdf}
\includegraphics[width=\textwidth]{figures/MOMI_fixed_normalized_worst_case.pdf}
\section{Conclusion}
- Cool and good

\section{Appendix}
\subsection{Data generation}\label{data}



\end{document}